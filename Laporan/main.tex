\documentclass[12pt,a4paper]{article}

% Encoding & language
\usepackage[utf8]{inputenc}
\usepackage[T1]{fontenc}
\usepackage[main=english, indonesian]{babel}
\usepackage{caption}
\usepackage{booktabs}

% untuk kode
\usepackage{listings}

% Margin & spacing
\usepackage{geometry}
\geometry{margin=1in}
\usepackage{setspace}
\setstretch{1.2}

% Math & extras
\usepackage{enumitem}
\usepackage{amsmath,amssymb,amsthm}
\usepackage{algorithm}
\usepackage{algpseudocode}

\theoremstyle{remark}
\newtheorem*{solusi}{Solusi}

% Paket untuk gambar
\usepackage{graphicx}

% Etc
\usepackage{titling}
\usepackage{csquotes}
% Untuk hyperlink
\usepackage{hyperref}

% --- Biblatex dengan IEEE style ---
\usepackage[backend=biber,style=ieee,language=english]{biblatex}
\addbibresource{refs.bib}

\DefineBibliographyStrings{english}{%
  references   = {Daftar Pustaka},
  bibliography = {Daftar Pustaka},
  and          = {dan},
  andothers    = {dkk.},
  pages        = {hlm.},
}

% STYLING
\setlength{\parindent}{2em}

% OVERRIDE Caption
\captionsetup[figure]{labelfont=bf, name=Gambar, justification=centering}
\captionsetup[table]{labelfont=bf, name=Tabel, justification=centering}



% --- Title page ---
\title{%
  \textbf{Tugas Mata Kuliah} \\
  \large Rekayasa Fitur dan Pengenalan Pola (RFPP) \\
  \large Klasifikasi Melanoma dengan Pendekatan \textit{Hand-crafted Features} dan Pemelajaran Mesin}
  
\author{Khalilullah Al Faath \\ NIU: 566643 \\ Magister Ilmu Komputer \\ [1cm]
  \includegraphics[width=0.4\textwidth]{images/logo-ugm.png}}
\date{Universitas Gadjah Mada \\ Semester Gasal 2025/2026}

\makeatletter
\renewcommand{\maketitle}{%
  \begin{titlepage}
    \centering
    \vspace*{2cm}
    
    {\LARGE \@title \par}
    \vspace{2cm}
    
    {\large \@author \par}
    
    \vfill
    
    {\large \@date \par}
  \end{titlepage}
}
\makeatother

\begin{document}

\maketitle
\thispagestyle{empty}

% Daftar isi dan daftar gambar/tabel

% override label
\renewcommand{\contentsname}{Daftar isi}



\renewcommand{\listfigurename}{Daftar Gambar}



\renewcommand{\listtablename}{Daftar Tabel}

% daftar isi
\tableofcontents

\newpage

% daftar gambar
\listoffigures

\newpage

% daftar tabel
\listoftables

\newpage
\setcounter{page}{1}

\section{Pendahuluan}

\section{Metode}

% \begin{figure}[H]
%     \centering
%     \includegraphics[width=1\linewidth]{images/diagram alir.pdf}
%     \caption{Diagram alir metode yang diusulkan.}
%     \label{fig:diagramalir}
% \end{figure}


% \begin{table}[H]
%     \centering
%     \begin{tabular}{c|c|c}
%         No. & Kelas & Jumlah data \\
%         1 & Maju & 22 \\
%         2 & Mundur & 26 \\
%         3 & Berhenti & 23 \\
%         4 & Kiri & 25 \\
%         5 & Kanan & 26 \\
%     \end{tabular}
%     \caption{Banyaknya data tiap kelas.}
%     \label{tab:rincianjumlahdata}
% \end{table}

\section{Hasil dan Pembahasan}

\subsection{Dataset yang digunakan}

\subsection{Lingkungan pengujian}

\subsection{Hasil}

\subsection{Diskusi}

\section{Kesimpulan}

Berdasarkan hasil dan analisis yang telah dilakukan, dapat disimpulkan bahwa:

Kode tersedia pada \textit{repository} GitHub \url{https://github.com/khalilullahalfaath/Automatic-Speech-Recognition}.

\newpage

% Daftar pustaka
\printbibliography

\end{document}
